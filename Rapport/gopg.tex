\title{Godkendelsesopgave - Overs\ae ttere 2011}

\author{
        Andreas Asp Bock: 061290-2917\\
        Mads R\o nnow: 151088-2101\\
        Daniel Str\o m: 1231231231
        }
\date{\today}

\documentclass[12pt,a4paper,danish]{article}
\usepackage[danish]{babel}
\usepackage[utf8]{inputenc}
%\usepackage{multicols}
%\usepackage{hyperref}
\usepackage{multicol}
\usepackage{longtable}
\usepackage{graphicx}
\usepackage{subfloat}
\usepackage{moreverb}
\usepackage{ulem}
\usepackage{multirow}
\usepackage{multicol}
\usepackage{tikz-inet}
\usepackage{graphicx}
\usepackage[hmargin={1in,1in}]{geometry}
\usepackage{color}
\usetikzlibrary{matrix}


\usepackage{slashbox}
\usepackage{amsmath}
\DeclareGraphicsExtensions{.pdf, .png, .jpg}
\usepackage{blindtext}
\usepackage{geometry}

%%%%%%%%%%%%%%%%%%
\begin{document} %
\maketitle       %
%%%%%%%%%%%%%%%%%%

%%%%%%%%%%%%%%%%%%%%%%%%%%%%%%%%%%%%%%
\renewcommand\abstractname{Abstract}%%
%%%%%%%%%%%%%%%%%%%%%%%%%%%%%%%%%%%%%%

\begin{abstract}
Følgende rapport udgør gruppens svar på Godkendelsesopgaven i kurset Oversættere i studieåret 2011/2012.
\end{abstract}

%%%%%%%%%%%%%%%%%%%%%%%%
\section*{Introduction}%
%%%%%%%%%%%%%%%%%%%%%%%%

Først vil vi gennemgå hver af de filer, vi har redigeret i, med henblik på at forklare hvilke ændringer vi har måtte lave for at kunne oversætte sproget 100.
Derefter...

Slutteligt vil vi komme med en sammenfatning og ...

% GENERELT:
% 	Fungerer det som det skal?
%	Hvorfor? Hvorfor ikke?
%	Inkluder store ændringer i programteskt i rapporten
%	Dokumenter kendte mangler

%%%%%%%%%%%%%%%%%%%
\section*{Lexeren}%
%%%%%%%%%%%%%%%%%%%

Vi har tilføjet de manglende tokens i lexerens \texttt{parse} funktion. Disse består af \texttt{while}-løkker, blokker (\texttt{\{ }, \texttt{\} }), paranteser, sammenligningsoperatoren,  enkelttegn (\texttt{charConst}) strenge (\texttt{stringConst}) samt referencer (pointere). Da \texttt{while} er et keyword har vi implementeret det ved at lave endnu en case i funktionen \texttt{keyword} i headeren.
Hvorfor dette fungerer ser på i næste sektion om parseren.

%%%%%%%%%%%%%%%%%%%%
\section*{Parseren}%
%%%%%%%%%%%%%%%%%%%%
% Hvordan er grammatikken gjort entydig?
% Beskrivelse af ikke-åbenlyse løsninger
% Opbygning af abstrakt syntax
% MosML-yacc må ikke rapportere konflikter

%%%%%%%%%%%%%%%%%%%%%%%%%
\section*{Type-chekeren}%
%%%%%%%%%%%%%%%%%%%%%%%%%
% Hvordan checker vi typer?
% figur 6.2, 7.3!

%%%%%%%%%%%%%%%%%%%%
\section*{Compiler}%
%%%%%%%%%%%%%%%%%%%%
%

%%%%%%%%%%%%%%%%%%%%%%%%%%
\section*{Fejlhåndtering}%
%%%%%%%%%%%%%%%%%%%%%%%%%%

%%%%%%%%%%%%%%%%%
\section*{Tests}%
%%%%%%%%%%%%%%%%%
% Udtømmende?



%%%%%%%%%%%%%%%%%%%%%%
\section*{Konklusion}%
%%%%%%%%%%%%%%%%%%%%%%




\bibliographystyle{abbrv}
\bibliography{main}
\end{document}


\begin{figure}[h!]
\begin{center}
\includegraphics[scale=0.4]{/home/andreas/LaTeX_Pics/OR/Assig2/Output(24).png} 
\caption{\textit{R} output for the transportation problem.} 
\end{center}
\end{figure}







